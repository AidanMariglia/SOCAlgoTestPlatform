\documentclass{article}

\usepackage{tabularx}
\usepackage{booktabs}
\usepackage{color}
\usepackage[normalem]{ulem}

\title{Problem Statement and Goals\\\progname}

\author{\authname}

\date{September 24, 2024}

%% Common Parts

\newcommand{\progname}{Software Engineering}
\newcommand{\authname}{Team 1, BANDwidth
\\ Declan Young
\\ Ben Dubois
\\ Nathan Uy
\\ Aidan Mariglia}                

\usepackage{hyperref}
    \hypersetup{colorlinks=true, linkcolor=blue, citecolor=blue, filecolor=blue,
                urlcolor=blue, unicode=false}
    \urlstyle{same}


\begin{document}

\maketitle

\begin{table}[hp]
\caption{Revision History} \label{TblRevisionHistory}
\begin{tabularx}{\textwidth}{lp{2cm}lp{3cm}lp{6cm}}
\toprule
\textbf{Date} & \textbf{Developer(s)} & \textbf{Change}\\
\midrule
September 24, 2024 & Declan Benjamin Nathan Aidan & Completed problem statement and goals\\
April 02, 2025 & Declan Benjamin Nathan Aidan & Revised problem statement and goals\\
\bottomrule
\end{tabularx}
\end{table}

\section{Problem Statement}

\subsection{Problem}

Battery state of charge (SOC) estimation is challenging, requiring specialized algorithms. Standardized testing is necessary to determine the accuracy and performance of proposed approaches. \sout{Building off of an existing solution for running SOC algorithms against a test suite, a cloud-based platform will be developed to streamline accepting and running user-submitted algorithms}. The current solution \textcolor{red}{that runs SOC algorithms against a test suite} has a convoluted submission process, which fails regularly, and executes the tests in serial. 

\subsection{Inputs and Outputs}

\textbf{Inputs:} \newline
User developed algorithms, either as matlab \sout{or python} code. \newline \newline
\textbf{Outputs:} \newline
Algorithm performance or resulting error, history of user submissions, rankings of algorithm performance.

\subsection{Stakeholders}

\begin{itemize}
    \item Dr. spencer smith
    \item Dr. Phil Kollmeyer
    \item Developers of the project
    \item Battery Engineering Research students
    \item Students taking battery engineering class
\end{itemize}

\subsection{Environment}
\textcolor{red}{The system requires both software and hardware components to function effectively. The following sections describe the specific software and hardware requirements.  }

\subsubsection{Software}
The software environment of the system will be Docker containers built for each component.

\subsubsection{Hardware}
The docker containers will be run on \sout{AWS, either ECS or EKS} \textcolor{red}{a local computer}. The hardware details of this service are abstracted away from the user.

\section{Goals}

\begin{itemize}
    \item The final product can test multiple algorithm submissions in parallel - since each algorithm takes approximately an hour to complete, running them in parallel will help reduce the overall time required to run all algorithms.
    \item \textcolor{red}{The final product has a user-friendly interface that simplifies the submission process and testing of algorithms.}
    \item The final product generates reports and compares algorithm performance \sout{- the reports will summarize the performance of the submitted algorithm and compare it with other submitted algorithms}, helping users understand and visualize their results.
    \item The final product can handle any error encountered throughout the algorithm’s runtime - this will minimize the software’s crashes and downtime and overall, it will increase the reliability of the software. It will make debugging much easier and as a result, improve the user experience.
    \item The final product is secure and prevents malware attacks - secure software is necessary to protect users’ \sout{sensitive data} \textcolor{red}{submissions and personal information} as well as keep the software’s integrity and availability.
\end{itemize}

\section{Stretch Goals}

Python Implementation of SOC Estimation Algorithm Testing Tool: Currently this tool is Matlab-based and this programming language is more niche and is mainly used in academic roles. For future development, it would be beneficial to have the testing tool and backend written in a programming language like Python that is widely known and used. Also, Matlab is not open-source and requires a license to be used, so translating to the open-source programming language of Python will allow this testing tool to be accessible to everyone.  

\section{Challenge Level and Extras}

The expected challenge level for this project is general. This challenge level is justified firstly by the fact that the technologies and domain knowledge required based on the expected implementation are relatively simple, especially when considering the experience that our group has with web applications and cloud development. Additionally, the problem being solved is not particularly novel, since although the problem itself isn’t something that is very commonly solved, the requirements/expected solution of a \sout{cloud-based} web application is something that is a very common solution to problems in the software industry. \newline \newline
As an extra for the project we will include thorough user guides/walkthroughs as a part of the final webapp, similar to the way github presents their tutorials. \textcolor{red}{Also, a hosting cost analysis will be included that compares different hosting providers.}


\newpage{}

\section*{Appendix --- Reflection}

\subsection*{Nathan Uy}

\begin{enumerate}
    \item Writing the problem statement allowed me to understand the scope of the project and its impact on stakeholders.
    \item Coming up with the extras for the project was challenging as it needed to be both useful and fun. So it required going through a lot of options and evaluating each of them.
\end{enumerate}  

\subsection*{Declan Young}

\begin{enumerate}
    \item When writing this deliverable I thought that our group did a great job on collaborating to agree on what the problem that we are solving is, so that the details could be described in a clear and concise way. Since the rest of this deliverable depended on and was based on the problem, it was crucial that we all had the same understanding of the problem.
    \item The main pain point we encountered for this deliverable was determining what stretch goals would be most suitable for our project. We resolved this by brainstorming as a group to create a list of possible stretch goals, from which we picked the goal that seemed most reasonable while staying interesting and closely related to the problem being solved.
\end{enumerate}  

\subsection*{Aidan Mariglia}

\begin{enumerate}
    \item Defining the problem statement happened quite easily while writing this deliverable. The existing project pitch, along with the information we gained while meeting with Dr. Kollmeyer provided us the necessary background to define the problem well.
    \item Determining stretch goals/extras was the worst sticking point of this deliverable. The existing problem was well contained, which added to this difficulty.
\end{enumerate}  

\subsection*{Benjamin Dubois}

\begin{enumerate}
    \item During this deliverable, I found that the team was on the same page for all of the required fields and this helped us to complete the deliverable as efficiently as possible. When brainstorming goals and the problem statement, we were able to make decisions very quickly as we all had the same idea of the problem and main goals. 
    \item The main pain point that we encountered during this deliverable was determining the stretch goals and extras. This was difficult as some features are hard to determine how long they will take at this moment, so it is difficult to know which features we can easily include in the timeline and which ones will need to be stretch goals. To resolve this, we created a list of the major development features we wanted to complete, and any features that seemed less significant were added to stretch goals or extras. 
\end{enumerate}  

\subsection*{Team}

\begin{enumerate}
\setcounter{enumi}{2}
    \item We adjusted our project’s scope to make it suitable for a Capstone project by firstly determining the scope of that project required to complete the project at the most basic level. We then added to this scope to make the project more complex and the scope more broad, until a point was reached where we felt that any more additions would be overwhelming and difficult to complete with the given time and resources. We increased the scope by adding extras to our project, as well as adding stretch goals that we hope to complete.
 
\end{enumerate} 

\end{document}